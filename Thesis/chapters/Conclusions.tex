\chapter{Conclusions}

This master's thesis has provided a comprehensive exploration of eBPF, a transformative technology reshaping the landscape of networking, observability and security in modern computing. 
Our study encompassed an in-depth examination of its historical evolution, intricate toolchain, versatile ecosystem and development aspects, with a particular focus on its presence and impact across both the Linux and Windows platforms.

The historical evolution of eBPF was traced, beginning with its origins in BPF and its subsequent evolution into eBPF. 
This journey highlighted how eBPF's flexible design enabled it to go beyond its initial networking-centric role, finding applications in diverse areas such as observability and security.

We delved into the complexities of the eBPF toolchain, revealing a sophisticated process that involves program development in languages like C, followed by compilation, verification and JIT compilation into native kernel code.
This marked eBPF's power and efficiency as a tool for extending kernel capabilities.

Then, the eBPF ecosystem was explored, emphasizing the significance of key components like maps for efficient data exchange and helpers for smooth interaction with the kernel.
These elements collectively constitute a robust toolkit for eBPF program development.
Our investigation into this topic highlighted its event-driven nature, allowing for the interception of system calls, function tracing and network traffic analysis, all without the need for kernel recompilation.

Furthermore, it is important to note that eBPF technology is still undergoing significant development. 
Moreover, its adoption and synchronization between Linux and Windows platforms are not fully aligned, as one platform's implementation preceded the other.

In the Linux community, we anticipate accelerated development of new eBPF tools, libraries and frameworks, expanding its utility into domains like IoT and edge computing.

Within the Windows domain, eBPF is at the edge of becoming widely adopted and essential in various software applications, particularly in areas like network monitoring, security enforcement, and performance optimization because more and more projects are recognizing its potential and starting to incorporate it into their solutions.
We also expect additional helpers, maps and hook points to be introduced in the near future to bridge the gap with Linux.

In summation, eBPF represents a remarkable evolution in systems programming, fundamentally altering our interaction with and understanding of modern computing environments. 
As eBPF continues to evolve, it is certain that it will play an increasingly crucial role in shaping the future of networking, observability and security across both Linux and Windows platforms. 
The boundless possibilities and the collective creativity of the community ensure an exciting trajectory ahead.

This thesis has been a journey of exploration and discovery and we extend our gratitude to all who have joined us on this intellectual voyage. 
As we conclude, we encourage everyone to embrace the profound potential that eBPF offers and to continue exploring and exploiting the capabilities of this remarkable technology in the years to come.
