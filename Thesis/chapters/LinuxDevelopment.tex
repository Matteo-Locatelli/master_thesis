\chapter{Linux development}

In the previous chapters we went through the evolution of eBPF throughout the years and we analyzed all the components of its ecosystem.
Now we are ready to jump into some coding and write out first eBPF program.

We are going to start to talk about the development process on Linux, since historically it was the first operating system where eBPF was introduced and there is a greater and more complete documentation.
In fact, on the internet there are various tutorials and guides on writing your first eBPF program: however, we are going to present just a couple of projects that in our opinion are the best for starting with eBPF because they set up as much things as possible for beginner users to let them dive straight into writing BPF programs and not get frustrated with various initials setup tasks.

Previously all from command line, but now there is the introduction of projects.

\section{Bumblebee}

% https://github.com/solo-io/bumblebee

Create example to help develop in future.

\section{libbpf-bootstrap}

% https://github.com/libbpf/libbpf-bootstrap/tree/master

% https://nakryiko.com/posts/libbpf-bootstrap/

% https://www.grant.pizza/blog/vmlinux-header/

% https://blog.aquasec.com/vmlinux.h-ebpf-programs