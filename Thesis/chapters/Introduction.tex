\chapter{Introduction}

In the ever-evolving landscape of computer science and networking, the demand for efficient, flexible and secure packet filtering technologies has been dominant. 
The \textit{Berkeley Packet Filter} (\textit{BPF}), an innovative technology developed in the Unix environment, has emerged as a powerful tool for network monitoring, traffic analysis and security enforcement. 
Over the years, BPF has undergone significant advancements, culminating in the birth of \textit{Extended Berkeley Packet Filter} (\textit{eBPF}), a groundbreaking extension that has revolutionized network engineering and performance analysis.

\section{Background}

Computer networks establish the backbone of modern communication, enabling the consistent exchange of information across the globe. 

The rapid growth of network traffic, the rise of complex cyber threats and the increasing need for real-time monitoring have motivated researchers and engineers to explore innovative solutions to enhance network performance and to build robust security mechanisms.
Packet filtering, a fundamental networking technique, serves as a first line of defense in safeguarding networks and optimizing data transmission.

Originally conceived in the 1990s, the Berkeley Packet Filter was designed as a mechanism to filter packets at the kernel level for the \textit{Berkeley Software Distribution} (\textit{BSD}) operating system, a discontinued operating system based on the early versions of Unix. 
However, its potential, consisting of its lightweight and versatile design, far exceeded its initial purpose and it evolved into a versatile technology with applications across various networking domains.

Over the years, BPF has undergone significant developments and adaptations, until it resulted in the advent of eBPF: with the introduction of a new virtual machine and bytecode, eBPF allowed for the dynamic execution of custom programs within the kernel context, extending its applicability beyond traditional packet filtering to areas such as network monitoring, tracing and deep packet inspection.

\section{Motivation}

Despite the extensive use of eBPF in Unix-based systems, its incorporation into Windows environments has remained a challenge. 
As Windows continues to be a prominent operating system in both personal and enterprise computing, unlocking the potential of eBPF on this platform becomes crucial for achieving cross-platform network engineering and security solutions.

This thesis will focus on the historical progression of BPF and its adaptation on the Windows platform.
In addition to that, we will explore the advancements introduced to eBPF on both operative systems and study the current state of art of eBPF on Windows to show its differences with the Linux environment. 

\section{Objectives}

This master's thesis aims to provide an in-depth analysis of eBPF's architecture, installation and functionality in both operating systems, while showing the history, development and impact of eBPF in the world of computer science and network engineering. 

The primary objectives of this research are as follows:

\begin{itemize}
	\item 
		Tell the history of eBPF: by understanding the origins of BPF, we gain insights into the motivations that led to the creation of eBPF and we can identify the key challenges faced during its integration into Windows and the innovative solutions designed to overcome them. 
		A look into the historical context provides a solid foundation for exploring eBPF's potential, from a simple packet filtering mechanism to a versatile technology with broader network real-world applications;
	\item 
		Installation and integration of eBPF on Linux and Windows: we will investigate the process of installing eBPF into both Linux and Windows operating systems. 
		By understanding the differences in installation procedure and requirements on these platforms, we are enabled to leverage the cross-platform capabilities of this technology;
	\item 
		Development of eBPF programs on Linux and Windows: this thesis will cover the development process of eBPF programs on both Linux and Windows platforms. 
		We will explore the process of creating, loading and executing eBPF programs.
		Furthermore, by studying the eBPF API, we will:
		\begin{itemize}
			\item 
				Demonstrate the creation of custom programs to achieve specific networking tasks;
			\item 
				Show how far they have come in the development of the technology in the two operating systems;
			\item 
				Examine the methods used to safely load eBPF programs into the kernel. 
		\end{itemize}
\end{itemize}

\section{Organization of the Thesis}

The subsequent chapters of this thesis will be organized as follows:

\begin{itemize}
	\item 
		Chapter 2 delves into the roots of eBPF, tracing its evolution from its inception to its current state;
	\item 
		In chapter 3, we dive deep into the technical structure of eBPF. 
		We explore its inner workings, focusing on its unique design and architecture.
		This chapter serves as a foundation for the practical applications discussed in subsequent chapters;
	\item 
		Chapter 4 explores the rich ecosystem surrounding eBPF. 
		We discuss essential tools like \textit{BCC} and \textit{libbpf}.
		This chapter showcases how eBPF is more than just a concept, but it is a thriving ecosystem;
	\item 
		Chapter 5 is dedicated to present the process of setting up eBPF and developing programs tailored in the Linux environments. 
		We will show some practical examples to provide a hands-on approach to mastering eBPF on Linux;
	\item 
		This chapter expands our scope to eBPF on the Windows platform. 
		We explore the recent integration of eBPF into the Windows ecosystem, offering step-by-step instructions for installation and program development. 
		Our goal is to bridge the gap between eBPF and Windows, making it accessible to a wider audience;
	\item 
		In the final chapter, we reflect on our journey through the world of eBPF. 
		We summarize key outcomes, discuss the current state of eBPF and explore its future prospects on both Linux and Windows. 
		This chapter serves as a culmination of our exploration and provides a forward-looking perspective.
\end{itemize}

Through this master's thesis, we hope to offer a comprehensive understanding of eBPF's significance, capabilities and potential in modern networking environments.
We also have the ambition to contribute to the field of computer engineering by closing the gap between Unix and Windows-based network technologies and security measures.
By exploring the installation and development processes on both Linux and Windows, we present a comparative analysis of eBPF's cross-platform capabilities.

\section{Repository of the project}

\textit{GitHub} is a platform and cloud-based service for software development and collaborative version control using \textit{Git}, a distributed version control system that tracks changes in any set of computer files, allowing developers to store and manage their code, owned by the company \textit{GitHub Incorporation}, whose logo is displayed in Figure \ref{fig:GitHub_logo}.
It provides the distributed version control of Git plus access control, bug tracking, software feature requests, task management, continuous integration and wikis for every project.
It is commonly used to host open source software development projects.

\begin{figure}[h]
	\centering
	\includegraphics[width=0.7\linewidth]{images/Technologies/GitHub_logo.png}
	\caption{GitHub \textit{Invertocat} logo \cite{GitHubLogo}.}
	\label{fig:GitHub_logo}
\end{figure}

Throughout the course of this master's thesis about eBPF, GitHub was an indispensable platform that played a dual role in enhancing our research journey. 

Firstly, it served as an efficient instrument to share the progress of the work with the co-advisors and made the collaboration during the entire development process easier. 
Its version control system allowed us to keep track of changes, maintain a detailed history of my project and collaborate consistently with the co-advisors, ensuring a smooth and efficient development workflow. 
By regularly pushing updates to the repository of this project \cite{MasterThesisRepo}, the co-advisors were able to monitor the evolution of the work, review code changes, provide timely feedback and offer valuable suggestions for improvement.

Secondly, GitHub was used as an invaluable resource for the eBPF community: during our research, we encountered several repositories (which we will discuss later) dedicated to developing and optimizing eBPF environments, tools and libraries. 
By studying and understanding their implementations, we were able to build upon the expertise and contributions of the open-source community, so that the quality and scope of our research have been enriched.

The open-source spirit of GitHub made knowledge exchange and collective growth easier, enabling us to contribute to the eBPF community while benefiting from the collective expertise it had to offer.
In fact, the public visibility of the GitHub repository of this project opens up the possibility of sharing our work with the wider community. 
By making the repository public, we hope that others can benefit from the knowledge and insights gained during the project, encouraging collaboration and contributions from future researchers and developers in the field of eBPF and its applications.
