\section*{Acknowledgements}

Completing this master thesis on the evolution and adaptation of Berkeley Packet Filter on both Linux and Windows platforms has been an enriching journey for me. 
I am deeply grateful to the individuals whose guidance, encouragement and support have made this research possible. 
Without their firm belief in my abilities, this effort would not have come to completion.

First and foremost, I extend my heartfelt gratitude to my esteemed advisor, professor Stefano Paraboschi, whose expertise, mentorship and invaluable feedback have been instrumental in shaping this thesis. 

My sincere appreciation must be extended to the people on the \textit{Unibg Security Lab} \cite{UnibgSeclabWebsite} team who actively participated in the development of this thesis: their continuous support throughout the entire research process have motivated me to push my boundaries and aim for excellence. 
I am grateful for their patience, insightful discussions and profound knowledge in the fields of computer engineering and systems security, which have significantly contributed to the depth and quality of this work: their willingness to share their expertise has been essential in overcoming various challenges faced during this study and in refining the ideas presented in this research.
Their support has made this academic pursuit not only a productive venture, but also an enjoyable one.
Also, they provided me with the LaTex template that I used to write this thesis \cite{UnibgLatexTemp}.

Speaking of people that gave me something practical to work on this project, I have to thank \textit{Subconscious Compute} \cite{SubComWebsite}, a company that decided to open-source a GitHub repository that has been fundamental to develop eBPF programs on the Windows platform and allowed me to do a better comparison with eBPF on the Linux environment, which was the scope of my master's thesis.
The availability of the repository not only provided me with a lot of resources and code examples, but also allowed me to gain insights into best practices and advanced techniques in programming with eBPF on Windows. 

Moreover, I would like to express my obligation to the wider academic community of the \textit{Università degli Studi di Bergamo} \cite{UnibgWebsite} for providing an environment that encourages learning, curiosity and innovation. 
I am deeply grateful to everyone who played a part, big or small, in the ending of my academic journey. 
The education I received has been invaluable and I am lucky to have had such exceptional guidance throughout my academic period. 
This thesis stands as a testament to the collective effort and support of those who have been part of my academic journey. 
The knowledge and experiences I have gained throughout the last five years have been instrumental in shaping my growth as a computer engineering student.

Last but not least, I must express my very profound gratitude to my parents for their love, encouragement and support throughout eighteen years of education. 
Their belief in my capabilities and constant motivation have been the driving force of my academic achievements, especially during the demanding period of writing this thesis.
I owe my successes to them because with their sacrifices they have allowed me to focus on my studies and achieve my academic goals, celebrating every milestone with infinite joy and pride.
In conclusion, I am grateful for the life lessons and values they instilled in me, which have shaped me into the person I am today.

Thank you.